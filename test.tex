\documentclass[conference]{IEEEtran}
\usepackage{cite}
\begin{document}
\title{System Architecture for a Class of Small Service Vehicle}
\author{\IEEEauthorblockN{Michael Shell\IEEEauthorrefmark{1},
	Homer Simpson\IEEEauthorrefmark{2},
	James Kirk\IEEEauthorrefmark{3}, 
	Montgomery Scott\IEEEauthorrefmark{3} and
		Eldon Tyrell\IEEEauthorrefmark{4}}
\IEEEauthorblockA{\IEEEauthorrefmark{1}School of Electrical and Computer Engineering\\
	Georgia Institute of Technology,
Atlanta, Georgia 30332--0250\\ Email: see http://www.michaelshell.org/contact.html}
\IEEEauthorblockA{\IEEEauthorrefmark{2}Twentieth Century Fox, Springfield, USA\\
	Email: homer@thesimpsons.com}
\IEEEauthorblockA{\IEEEauthorrefmark{3}Starfleet Academy, San Francisco, California 96678-2391\\
	Telephone: (800) 555--1212, Fax: (888) 555--1212}
\IEEEauthorblockA{\IEEEauthorrefmark{4}Tyrell Inc., 123 Replicant Street, Los Angeles, California 90210--4321}}
\maketitle

\begin{abstract}
By the dawn of the new millennium, robotics has undergone a major transformation scope and dimensions. This expansion has been brought about by the maturity of field and the advances in its related technologies. From a largely dominant focus, robotics has been rapidly expanding into the challenges of the world. The recent technology development demonstrated how the fast growing progress toward the of new perception, control, and motion planning techniques allow autonomous vehicles not only to travel significant distances in off-road, but also to operate in urban scenarios. The expected impact in the commercial sector for automotive manufacturers is equally if not more: autonomous sensing and control constitute key technologies to vehicles the future that might help save thousands of lives now lost in traffic accidents.

Imagine a large field of vehicles without drivers traversing 60 miles in live, operating entirely without human guidance. A complex course including, traffic circles, and parking lots, defined by just kilobytes of data, several meters wide traveling down lanes only slightly wider, using systems with an accuracy of several meters. Humans in full-size unmanned vehicles at speeds up to 60 miles per hour. This actually would become reality in the near future.

This project aims to bring the recent development on autonomous driving technology into a more practical and marketable platform so that it benefits can really be advantageous to real users \cite{mrx05,IEEEexample:bluebookstandard,IEEEexample:softonline,Dellaert06ijrr}.
\end{abstract}
%,./IEEEtranBST/IEEEexample.bib

\bibliographystyle{IEEEtran}
\bibliography{IEEEabrv,./IEEEtranBST/IEEEexample,isam}
\end{document}
