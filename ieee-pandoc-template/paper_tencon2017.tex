

\newcommand{\CLASSINPUTtoptextmargin}{104pt}
\newcommand{\CLASSINPUTbottomtextmargin}{54pt}
\newcommand{\CLASSINPUToutersidemargin}{37pt}
\newcommand{\CLASSINPUTinnersidemargin}{54pt}

%\renewcommand{\IEEEtitletopspaceextra}{18pt}
%\documentclass[10 pt,a4paper,conference]{IEEEtran}
%\documentclass[10 pt,a4paper,conference]{IEEEconf}
\documentclass[10 pt,a4paper,conference]{IEEEtran}


\usepackage{amssymb,amsmath}
\usepackage{ifxetex,ifluatex}
\usepackage{fixltx2e} % provides \textsubscript

\ifnum 0\ifxetex 1\fi\ifluatex 1\fi=0 % if pdftex
  \usepackage[T1]{fontenc}
  \usepackage[utf8]{inputenc}
\else % if luatex or xelatex
  \ifxetex
    \usepackage{mathspec}
    \usepackage{xltxtra,xunicode}
  \else
    \usepackage{fontspec}
  \fi
  \defaultfontfeatures{Mapping=tex-text,Scale=MatchLowercase}
  \newcommand{\euro}{€}


\fi
% use upquote if available, for straight quotes in verbatim environments
\IfFileExists{upquote.sty}{\usepackage{upquote}}{}
% use microtype if available
\IfFileExists{microtype.sty}{%
\usepackage{microtype}
\UseMicrotypeSet[protrusion]{basicmath} % disable protrusion for tt fonts
}{}


\ifxetex
  \usepackage[setpagesize=false, % page size defined by xetex
              unicode=false, % unicode breaks when used with xetex
              xetex]{hyperref}
\else
  \usepackage[unicode=true]{hyperref}
\fi
\hypersetup{breaklinks=true,
            bookmarks=false,
            pdfauthor={},
            pdftitle={Human Tracking and Following in Dynamic Environment for Service Robots},
            colorlinks=false,
            citecolor=blue,
            urlcolor=blue,
            linkcolor=magenta,
            pdfborder={0 0 0}}
\urlstyle{same}  % don't use monospace font for urls
\usepackage[numbers,square]{natbib}
\bibliographystyle{IEEEtranN}


\usepackage{listings}

\usepackage{algorithm}
\usepackage{algorithmic}
%\usepackage{algpseudocode}
%\usepackage[]{algorithm2e}
\usepackage{pifont}


%\makeatletter
%\algrenewcommand\ALG@beginalgorithmic{\footnotesize}
%\makeatother

\usepackage{tabularx}
\usepackage{array}






\usepackage{graphicx,grffile}
\makeatletter
\def\maxwidth{\ifdim\Gin@nat@width>\linewidth\linewidth\else\Gin@nat@width\fi}
\def\maxheight{\ifdim\Gin@nat@height>\textheight\textheight\else\Gin@nat@height\fi}
\makeatother
% Scale images if necessary, so that they will not overflow the page
% margins by default, and it is still possible to overwrite the defaults
% using explicit options in \includegraphics[width, height, ...]{}
\setkeys{Gin}{width=\maxwidth,height=\maxheight,keepaspectratio}
\graphicspath{ {./figures/iros2016/} }



%\setlength{\parindent}{0pt}
\setlength{\parskip}{6pt plus 2pt minus 1pt}
\setlength{\emergencystretch}{3em}  % prevent overfull lines
\providecommand{\tightlist}{%
  \setlength{\itemsep}{0pt}\setlength{\parskip}{0pt}}
\setcounter{secnumdepth}{5}


%\renewcommand{\IEEEtitletopspaceextra}{18pt}
%\title{Human Tracking and Following in Dynamic Environment for Service Robots}
\title{\LARGE \bf Human Tracking and Following in Dynamic Environment for Service Robots}



\ifCLASSOPTIONcompsoc
\usepackage[caption=false,font=normalsize,labelfon
t=sf,textfont=sf]{subfig}
\else
\usepackage[caption=false,font=footnotesize]{subfi
g}
\fi

\author{
	\IEEEauthorblockN{
		Adiwahono, A. H.\IEEEauthorrefmark{1}, 
		Saputra, V. B.\IEEEauthorrefmark{2}, 
		Ng, K. P.\IEEEauthorrefmark{5}, 
		Gao, W.\IEEEauthorrefmark{7}, 
		Ren, Q.\IEEEauthorrefmark{6}, 
		Tan, B. H.\IEEEauthorrefmark{4}, 
		Chang, T.W.\IEEEauthorrefmark{3} 
		}
	\IEEEauthorblockA{%
		Institute for Infocomm Research \\
		1 Fusionopolis Way, Singapore 138632 \\
		Email: (\IEEEauthorrefmark{1}adiwahonoah  \IEEEauthorrefmark{2}saputravb  \IEEEauthorrefmark{5}ngkp  \IEEEauthorrefmark{7}gaowc  \IEEEauthorrefmark{6}renqy  \IEEEauthorrefmark{4}tanbh  \IEEEauthorrefmark{3}changtw  ) @i2r.a-star.edu.sg
	}
}


\date{}

\usepackage{color}

\definecolor{lightred}{RGB}{175,50,50}
\definecolor{lightgreen}{RGB}{0,150,0}
\definecolor{lightblue}{RGB}{50,50,175}

\lstdefinelanguage{diff}{
  morecomment=[f][\color{lightblue}]{diff },
  morecomment=[f][\color{lightblue}]{index },
  morecomment=[f][\color{lightblue}]{@@},     % group identifier
  morecomment=[f][\color{lightred}]-,         % deleted lines
  morecomment=[f][\color{lightgreen}]+,       % added lines
  morecomment=[f][\color{lightblue}]{---}, % Diff header lines (must appear after +,-)
  morecomment=[f][\color{lightblue}]{+++},
}
\hyphenation{}

\newcommand{\attn}[1]{{\color{red}#1}}
\newcommand{\desc}[1]{{\emph{\color{blue}#1}}}
\newcommand{\needcite}{\attn{\tiny{[cite]}}}
\newcommand{\todo}[1]{\colorbox{yellow}{\bf TODO: #1}}


\lstset{ %
  basicstyle=\ttfamily\footnotesize,        % the size of the fonts that are used for the code
  breakatwhitespace=false,         % sets if automatic breaks should only happen at whitespace
  breaklines=true,                 % sets automatic line breaking
  columns=fixed,
  keepspaces=true,                 % keeps spaces in text, useful for keeping indentation of code (possibly needs columns=flexible)
  numbers=left,                    % where to put the line-numbers; possible values are (none, left, right)
  numbersep=10pt,                   % how far the line-numbers are from the code
  numberstyle=\tiny, % the style that is used for the line-numbers
  showspaces=false,                % show spaces everywhere adding particular underscores; it overrides 'showstringspaces'
  showstringspaces=false,          % underline spaces within strings only
  showtabs=false,                  % show tabs within strings adding particular underscores
  stepnumber=1,                    % the step between two line-numbers. If it's 1, each line will be numbered
  tabsize=2,                       % sets default tabsize to 2 spaces
}

\urlstyle{rm}

% Redefines (sub)paragraphs to behave more like sections
%\ifx\paragraph\undefined\else
%\let\oldparagraph\paragraph
%\renewcommand{\paragraph}[1]{\oldparagraph{#1}\mbox{}}
%\fi
%\ifx\subparagraph\undefined\else
%\let\oldsubparagraph\subparagraph
%\renewcommand{\subparagraph}[1]{\oldsubparagraph{#1}\mbox{}}
%\fi

\begin{document}
\maketitle

\begin{abstract}
Robot human interactions have significant applications in dealing with
day-to-day robot service. This paper presents a novel approach for human
detection, tracking and following. A flexible and robust detection
method using a 2D laser scanner is proposed firstly. The detection
method can be adapted to various leg characteristics. The detection
result is then fed to a Kalman Filter based tracking algorithm. Owing to
the inherent ability of the embedded Kalman Filter, the tracking
algorithm is able to track specific persons and their motion. Then, a
controller is designed to follow target persons while avoiding
obstacles. Finally, a service robot, namely ISERA, is adopted in the
experiments for verifying the proposed approach. The experimental
results show the robot can following a target person both indoor and
outdoor dynamic environment.
\end{abstract}

%%\begin{IEEEkeywords}
%    %feature extraction;
%    %mobile robots;
%    %trajectory control;
%    %laser scanner;
%    %Path planning%\end{IEEEkeywords}
%


\section{Introduction}\label{introduction}

Mobile service robots, such as robotic butlers, robotic waiters, and
robotic pets, are designed to assist people in their daily lives. In
these robots, the capability to autonomously detect, track and follow a
person has been regarded as an important functionality of the robotic
systems. However, until now, it still remains a challenge that endowing
this capability to these robots when they operate in human environment.

The approaches of autonomous person following without the person having
to wear any active beacon can be classified into two categories: vision
based and laser based. Due to the low cost, RGB-D cameras are widely
adopted in the vision based approaches as the work presented in
\citep{ref0, ref1}. However, the minimum distance requirement and the
narrow field view of the RGB-D sensors limit the applications for
following persons.

The laser based approaches have an advantage in terms of large field of
view and robustness on any lighting condition. In \cite{ref3} the map
information is used to help tracking both moving and stationary person.
However, the usage of this approach will be limited on environments that
have been mapped. On the other hand, leg identification has been used
for people tracking in an unknown area (e.g.~in \cite{ref4} -
\cite{ref7}). More recently, \cite{ref7} introduces a framework for
outdoor person tracking and following for a smart wheel chair robot by
extending a trained legs detector from \cite{ref8}. In
\citep{ref9, ref10}, robotic behaviors in relation to the speed and
distance to the person being followed have been analyzed. However, the
correlation from detecting legs into tracking a person, and the
robustness when multiple people is around the person being followed has
not been explored in these studies.

\begin{figure}[!t]
\centering
\includegraphics[width=3.5in]{ISERA.pdf}
\caption{(a) ISERA robot in action of tracking and following a
target person (b) The sensor and interface of ISERA service
robot}\label{fig_isera}
\end{figure}

In this paper, we propose an approach to autonomously classify, track
and follow persons in a dynamic environment for a service robot, namely
ISERA (shown in Fig. \ref{fig_isera}). Using 2D Lidar and Odometry, our
approach does not require prior map information, and benefit from wide
field of view and consistent in any lighting condition. In our approach,
a constraint based leg detection method is developed to regulate the
strictness of the leg identifier as needed. Compared to previous works
that solely track the legs, our proposed approach associate the data to
form and track a virtual target person. In this way, brief false
detections of a leg in crowded environment would not immediately made
the person jump to another location, due to physical constraints in
relation to the other leg. Two layers of Kalman filter is used to ensure
the robustness of the person tracking despite close proximity to another
person and temporary line of sight loss to the leg clusters occurred.

Another notable merit of the proposed approach is that a real time
obstacle avoidance algorithm is embedded into the feed-back motion
controller of the robot. Hence, experiments show ISERA can follow a
target person while avoid obstacles autonomously in a cluttered
environment.

The paper is organized as follows. Section II describes the mobile robot
system. The people detection, tracking and following approach is
detailed in Section III. The experiment and results are described in
Section VI. Finally, a conclusion is given in Section V.

\section{System Description}\label{system-description}

The ISERA robot is designed as a human friendly service robot that could
serve as a butler or courier in public area. ISERA is built on the top
of a differential drive mobile platform with a maximum speed of
\(1m/s\). A PC is installed inside the robot for processing and a Hokuyo
UTM-30-LX lidar is mounted about \(30cm\) from the ground, which
corresponds to sensor reading of legs below the knee, around the shank
area of average adults. The lidar provide 2D laser scan, which is used to find groups of points that fits the criterions of clusters in the environment.

\section{Person Following}\label{person-following}

\subsection{Leg Identification and Segmentation}\label{leg-identification}

%\subsubsection{Leg Segmentation}\label{leg-segmentation}

%\paragraph{Modeling and Evaluation of
%clusters}\label{modeling-and-evaluation-of-clusters}

A leg identification that build based on the principle of circle
geometric fitting is proposed. To be classified as an arc or convex shape of a leg, the middle point of
a cluster must be within a reasonable distance ratio (typically between
0.1-0.7) from the length between the cluster edges. The clusters that
pass this filter will be processed further, otherwise it will be
discarded. This filtering quickly discards any non-convex shape cluster
with minimum computation cost.

Next, the center point of the cluster is approximated as the virtual
point in the middle of the cluster leftmost and rightmost edges. For
each cluster, the distance between the lidar and the center point of
that cluster is denoted as \(z\) and will be used as a variable in the
analysis. Another addition is the algorithm could also identify a joint dual leg, where the two legs of a person standing could be very close to each other and seen as one large cluster by the sensor. When a cluster has a local minimum that is farther then the center point of the cluster, it will be dissected at the minimum point into two parts. The part with more
sampling points will be considered as a new single leg cluster and will be processed further. 

%Next, the center point of the cluster is approximated as the virtual point in the middle of the cluster leftmost and rightmost edges. For each cluster, the distance between the lidar and the center point of that cluster is denoted as $z$ and will be used as a variable in the analysis. This is to facilitate the effect of distance into the perception data. 

%Another addition
%is the algorithm could also identify a joint dual leg, where the two
%legs of a person standing could be very close to each other and seen as
%one large cluster by the sensor.

To analyze the single leg clusters, three criterions are used as
benchmark. First, the mean error of the inscribe angle variant (IAV) of
the cluster \citep{DBLP:conf/icra/XavierPCRN05}. The second is the error
of the standard deviation of the cluster. The third is related to the
error in the size of the cluster.
The benchmark criterion is taken from numerous measurement of an adult
humanoid model with typical cylindrical legs at varying distance \(z\)
to the lidar. The data set collected are then interpolated into a
function of distance \(z\) and kept as a benchmark values.

Instead of giving a crisp output, our
algorithm calculates the confidence of a segmented cluster to be
considered as a leg. Based on the normalized errors between the desired benchmark and
the actual measurement, an equation to formulate the confidence of a
cluster to be considered as a leg is derived as,

\begin{equation}
Leg_{c}=max(1-K_II_N-K_SS_N-K_PP_N,0)\,. \label{eq:cost}
\end{equation}

with

\begin{align*}
I_N=(|I_c-I_d(z)|)/{I_d(z)}\, \\
S_N=(|S_c-S_d(z)|)/{S_d(z)}\, \\
P_N=(|P_c-P_d(z)|)/{P_d(z)}\, \\
\end{align*}

Where \(Leg_{c}\), which has a value between \(0\) and \(1\), is the
confidence of cluster \(c\) to be considered a leg. \(I_c\), \(S_c\),
and \(P_c\) are IAV mean, IAV standard deviation, and number of points
that are calculated from sensor data, respectively. \(I_d(z)\),
\(S_d(z)\),and \(P_d(z)\) are the desired values of the IAV mean, IAV
standard deviation, and number of points in the cluster for distance
\(z\), respectively. \(K_I\), \(K_S\), and \(K_P\) are weighting factor
rule that tuned based on the preferred strictness of the leg
segmentation.

Strict leg segmentation gives the least number of false positives, and
could be used when the lidar identifies mostly bare legs. On the other
hand, if the circumstance allows us to tolerate more false positives in
the leg detection, loose segmentation could be used to detect people
that are wearing loose pants. This strictness option would provide
situational advantage compared to a rigid data trained learning based
methods in other works. Finally, for clusters that have a confidence that exceed a cut off
threshold, the center point and the confidence value are passed on to
the leg and person tracking algorithm.

\subsection{Leg and People Tracking}\label{leg-and-people-tracking}

In order to follow certain person, multiple target tracking is developed
to estimate positions of both detected legs and people. The estimation
is implemented using Kalman Filters (KFs) \citep{KalmanRudolfEmil} with
two different motion model for legs and for people. In general, the
state estimation of the legs and people can be modeled by a linear
system,
\(\boldsymbol{\mathrm{x}}_{k+1}=\boldsymbol{A} \boldsymbol{\mathrm{x}}_k + \boldsymbol{\mathrm{w}}_k\).
The process noise, \(\boldsymbol{\mathrm{w}}_k\) is modelled as Gaussian
white noise with covariance matrix \(\boldsymbol{Q}\). Fig.
\ref{fig_process_diagram} shows the flow of the people following
procedure. The input for the legs tracking are the leg detection results
in each frame \(k\). Furthermore, the input for the people tracking is
the leg tracking states, treated as observations. In contrast with our
approach, temporary people tracking are created in each update step in
\citep{ref7}. This approach could infer people directly from leg
identification, but the movements of the people are not included in the
consideration of the filter. In our method, the KF in the people
tracking maintains the state of the tracked people at all times.
Finally, the result of the people tracking is transmitted to the pursuit
controller (Sec. \ref{pursuit-controller}), so that the robot can follow
the person.

\begin{figure}[!t]
\centering
\includegraphics[width=2.5in]{process_diagram.pdf}
\caption{Tracking process.}
\label{fig_process_diagram}
\end{figure}

\subsubsection{Legs Tracking}\label{legs-tracking}

The KF for the legs uses a constant velocity motion model with a pseudo
velocity measurement during the KF update steps. At discrete time \(k\),
the KF maintains a set of leg tracks,
\({}^L\boldsymbol{X}_k=\lbrace {}^L\boldsymbol{\mathrm{x}}^1_k, {}^L\boldsymbol{\mathrm{x}}^2_k,\dotsb,{}^L\boldsymbol{\mathrm{x}}^{{}^LN_k}_k \rbrace\),
where \({}^LN_k\) is the number of leg track at time \(k\). Each leg
track has a state estimate
\({}^L\boldsymbol{\mathrm{x}}^j_k=[ x \: y \: \dot{x} \: \dot{y} ]^T\)
of a target position and velocity in a 2D Cartesian coordinate. During
the update step of the Kalman Filter, leg identification are processed
using an observation model,
\({}^L\boldsymbol{\mathrm{z}}_k={}^L\boldsymbol{H}{}^L\boldsymbol{\mathrm{x}}_k + \boldsymbol{\mathrm{v}}_k\).
The observation includes position and velocity observations with white
noise, \(\boldsymbol{\mathrm{v}}_k\) governed by a covariance matrix,
\({}^L\boldsymbol{R}\). The pseudo velocity measurement is determined
from estimation of the difference from current state (after update step)
and the previous state, normalized by the time step. The velocity
measurement is used since we found that data from the laser perception
is accurate and helps to improve legs tracking accuracy. The prediction
and update cycle of the KF for every track can be summarized by the
following equation. For clarity, the indices are omitted in this
equation.

\begin{IEEEeqnarray}{c}
$$
\boldsymbol{\tilde{\mathrm{x}}}=\boldsymbol{A}\boldsymbol{\hat{\mathrm{x}}}\IEEEyesnumber\IEEEyessubnumber*\\
\boldsymbol{\tilde{P}}=\boldsymbol{A\hat{P}A}^T + \boldsymbol{Q}\\
\boldsymbol{K}=\boldsymbol{\tilde{P}H}^T(\boldsymbol{H\tilde{P}H}+ \boldsymbol{R})^{-1}\\
\boldsymbol{\hat{\mathrm{x}}}=\boldsymbol{\tilde{\mathrm{x}}}+\boldsymbol{K}(\boldsymbol{\mathrm{z}}-\boldsymbol{H\tilde{\mathrm{x}}})\\
\boldsymbol{\hat{P}}=(\boldsymbol{I}-\boldsymbol{KH})\boldsymbol{\tilde{P}},
$$
\end{IEEEeqnarray}

where \(\backsim\) and \(\wedge\) denote the predicted and filtered
quantities, \(\boldsymbol{P}\) is the covariance matrix,
\(\boldsymbol{H}\) is the observation matrix relating the state to the
observation, \(\boldsymbol{I}\) is the identity matrix and
\(\boldsymbol{K}\) is the Kalman gain. The update steps require
associations between predicted state and current observation, which is
solvable by employing Munkres algorithm \citep{munkresalgo}. The cost
matrix for the assignment problem is computed using the Mahalanobis
distance, so that the filter uncertainties are taken into account. This
method is also known as Global Nearest Neighbor. However, a global
gating threshold setting is employed so that association pairs that are
too far away will be dropped. The data association has three
interpretations: leg(s) that are uniquely associated with a track, the
legs that cannot be associated with any of the existing track(s), and
the track(s) that cannot be associated to any identified leg. Each track
has also confidence level which is increased exponentially (with a
constant \(\alpha\) as a parameter) if there is evidence. If there is no
evidence, the confidence will decrease and the track is removed if the
confidence is below certain threshold. The outcomes of the
interpretations are summarized in Table.
\ref{table:leg_tracking_outcome}. The \({}^Lc^j_k\) and \({}^Ld^j_k\)
are the confidence of the leg track and the identified leg confidence
associated with the leg track, with index \(j\) at time \(k\),
respectively.

\begin{table}[!t]
\renewcommand{\arraystretch}{1.3}
\caption{Possible outcomes for leg tracking association}
\label{table:leg_tracking_outcome}
\centering
\footnotesize
\begin{tabular}{p{1.5cm}||p{6.cm}}
  \hline
    \textbf{Case} &
    \textbf{Outcome} \cr
  \hline\hline
    Assigned track with leg observation & Update the corresponding KF state. Track confidence is updated, ${}^Lc^j_k=\alpha \, {}^Lc^j_{k-1} + (1-\alpha) \, {}^Ld^j_{k}$\cr
  \hline
    Unassociated track & Skip update, but propagate the track using the KF predict step. Track confidence is degraded, ${}^Lc^j_k=\alpha \, {}^Lc^j_{k-1}$ \cr
  \hline
    Unassociated leg observation & Initiate a new track with zero velocity and zero confidence. \cr
  \hline
\end{tabular}
\end{table}

\subsubsection{People Tracking}\label{people-tracking}

In order to track people, additional KFs are also created similarly with
the formulation elaborated in the leg tracking method. The differences
lie on the motion model, observation model and how the data association
is treated. People tracking maintains a set of people tracks,
\({}^P\boldsymbol{X}_k=\lbrace {}^P\boldsymbol{\mathrm{x}}^1_k, {}^P\boldsymbol{\mathrm{x}}^2_k,\dotsb,{}^P\boldsymbol{\mathrm{x}}^{{}^PN_k}_k \rbrace\),
where \({}^PN_k\) is the number of people track at time \(k\). Each leg
track has a state estimate
\({}^P\boldsymbol{\mathrm{x}}^j_k=[ x \: y \: \dot{x} \: \dot{y} \: \ddot{x} \: \ddot{y} ]^T\)
of a target position, velocity, and acceleration in a 2D Cartesian
coordinate. The rationale behind the constant acceleration model for the
people tracking is that humans have walking pattern that accelerates and
decelerates periodically. The observation model takes the states of the
legs tracks as measurements,
\({}^P\boldsymbol{\mathrm{z}}^m_k={}^L\boldsymbol{\mathrm{x}}^m_k\),
where \(m=1 \dotsb {}^LN_k\). With this another layer of KF tracker, any
spurious leg identification can be robustly filtered out.

The people tracking observation model,
\({}^P\boldsymbol{\mathrm{z}}_k={}^P\boldsymbol{H}{}^P\boldsymbol{\mathrm{x}}+\boldsymbol{\mathrm{v}}_k\)
takes only position as measurement variable. If both legs of a person
can be detected, the position is computed as the center of the two legs,
otherwise it is coincide as the position of the only leg detected. The
people associations is solved using the Munkres algorithm, but the
observation targets are firstly categorized into three groups: the legs
that has unambiguous nearest leg, the legs that has uncertain leg pair
(because there are a few candidates nearby), and the legs that are
located far enough to other legs. These groups are subsequently referred
as the two-legs group, the ambiguous one-legs group, and the certain
one-legs group, respectively. The grouping is determined based on simple
search with a maximum distance between two-legs pair threshold
parameter. The data association is solved using similar cost matrix
method, however we found that for people tracking, real distance metric
is more robust. This is possibly caused by the fact that the
uncertainties of the KFs are not as accurate compared to the leg
tracking, because the observation are virtual measurement (the state of
the leg tracks). There are four interpretations of the people tracking
association: two-legs group that are uniquely associated with a people
track, two-legs group that are unassociated, the people track(s) that
are paired with only one in the one-legs group, the people track(s) that
are unpaired with any observation. The summary of the outcomes for these
interpretations is shown in Table. \ref{table:people_tracking_outcome}.
People tracks carry additional confidence information derived from its
observation from leg tracking data. The confidence is update in a
similar manner as in the leg tracking, with a constant parameter
\(\beta\). Two-legs group confidence is computed by averaging its legs
confidences, and a single confidence is derived when a people track is
associated with the one-legs group. If associated with the uncertain
one-legs group, people track confidence are not updated, i.e.~maintain
the same confidence value as it is deemed ambiguous. The track will be
removed if the confidence is below certain threshold. The \({}^Pc^j_k\)
and \({}^Pd^j_k\) are the confidence of the people track and the
confidence of the associated observation, with index \(j\) at time
\(k\), respectively.

\begin{table}[!t]
\renewcommand{\arraystretch}{1.3}
\caption{Possible outcomes for people tracking association}
\label{table:people_tracking_outcome}
\centering
\footnotesize
\begin{tabular}{p{1.5cm}||p{6.cm}}
\hline
\textbf{Case} &
\textbf{Outcome} \cr
\hline\hline
Assigned people track with two-leg & Update the corresponding KF state. Track confidence is updated, ${}^Pc^j_k=\beta \, {}^Pc^j_{k-1} + (1-\beta) \, {}^Pd^j_{k}$\cr
\hline
Unassociated people track & Skip update, but propagate the track using the KF predict step. Track confidence is degraded, ${}^Pc^j_k=\beta \, {}^Pc^j_{k-1}$ \cr
\hline
Associated people track with one-leg & If the one-leg belongs to certain one-legs group, update the corresponding KF state and its confidence using the properties of the one-leg. \newline ,else if the one-leg is in the ambiguous group, the KF state is not updated, and the predict step is skipped for this track.\cr
\hline
Unassociated two-leg & Initiate a new track with zero velocity, zero acceleration and zero confidence. \cr
\hline
\end{tabular}
\end{table}

\subsection{Pursuit Controller}\label{pursuit-controller}

The pursuit controller drives ISERA towards the tracked target person
while avoiding obstacles. The driving control is based on the selection
of discrete spatial zones with the lowest cost. Fig.
\ref{fig_pursuit_controller} shows the variables used in the pursuit
controller. The coordinate of the target person is described in polar
coordinate with range \(R_P\), and angle \(\theta_{P}\) with respect to
the robot heading. To determine the angle \(\theta_{BP}\) where the
robot should go to pursuit the target, the area surrounding the robot
are first divided into \(m\) number of spatial `buckets'' zones. The
costs of each bucket are subject to evaluation, determined by factors
such as the angle of each bucket to the target and to obstacles.

\begin{figure}[!t]
\centering
\includegraphics[width=3.5in]{PursuitIllustration.pdf}
\caption{Variables in the pursuit controller. (a) Illustration of the discrete spatial zone into $m$
number of buckets (9 buckets for illustration purpose). Note that
the nearest point to obstacle $C_{Bn}$ in bucket $n=\{1,...,m\}$ may
vary up depending on the environment (b) The parameters used in the
pursuit controller} \label{fig_pursuit_controller}
\end{figure}

A dynamic lookout distance is used to regulate the obstacle detection
area where the lookout distance of each buckets \(R_{OD}\), are kept to
be less than the range to target. The area is shrunk and expanded
proportional to the distance to the target \(R_P\). In this way, the
target person legs would not be regarded as an obstacle to be avoided.
The delta angle of each bucket with respect to the target
\(\delta_{Bn}\) is calculated as,

\begin{equation}
\delta_{Bn} = |\theta_{Bn}-\theta_{P}|\,. \label{eq:delta}
\end{equation}

where \(\theta_{Bn}\) is the angle of the center point in bucket \(n\),
and \(\theta_{P}\) is the target angle. Then, the individual cost of
each bucket is calculated as,

\begin{equation}
cost_{Bn}=\frac{\delta_{Bn}+1}{C_{Bn}}\,. \label{eq:cost}
\end{equation}

where \(C_{Bn}\) is the distance to the nearest obstacle inside the
buckets.

Naturally, when there is no obstacle inside the obstacle detection
range, the nearest bucket in the direction of the target would have the
lowest cost. However, to prevent the robot from going too near into an
obstacle, each bucket cost is also influenced by the cost of its
neighboring several buckets. By comparing the average cost of each
buckets, the center point of the bucket with the lowest average cost is
chosen to be the pursuit heading \(\theta_{BP}\).

The turn velocity \(\omega_P\) is chosen to be proportional to pursuit
heading angle \(\theta_{BP}\),

\begin{equation}
\omega_P=K_{\omega}\times\theta_{BP}\,. \label{eq:cost}
\end{equation}

Where \(K_{\omega}\) is a weighting constant.

The pursuit velocity \(v_P\), is made to be proportional to obstacle
clearance in the pursuit heading \(C_{BP}\), the angle of the target
person \(\theta_P\), and the angle of the pursuit heading
\(\theta_{BP}\). For example, to pursuit with the highest speed, there
should be no obstacle near the pursuit heading and the target person is
in front of the robot. For other cases, the velocity will be reduced
proportionally.

Another factor that has a significant impact on the robot behavior is
the social zones of the person \cite{ref10}. As a service robot, the
robot is expected to behave in the manner that is socially acceptable.
The closer the robot to the target person, the more gentle the motion
should be. The control laws combining these considerations are,

\begin{figure*}[!t]
\centering
\subfloat[Single target tracking]{\includegraphics[height=3.0in]{occluded-cropped.pdf}
\label{fig_static_result_one_people}}
\subfloat[Two targets tracking]{\includegraphics[height=3.0in]{occluded2.pdf}
\label{fig_static_result_two_people}}
\hfil
\subfloat{\includegraphics[width=1.2in]{legend.pdf}}
\caption{Tracking results for stationary robot.}
\label{fig_static_result}
\end{figure*}

\begin{equation}
v_P=\frac{C_{BP}}{K_o}\times\frac{\theta_m-\theta_P}{\theta_m}\times
\frac{\theta_m-\theta_{BP}}{\theta_m}\times D_P\,.
\label{eq:forwardvel}
\end{equation}

Where \(K_o\) is a weighting constant, and \(\theta_m\) is an angle
between the front and the edge of the perception zone. \(D_P\) is the
parameter that changes the aggressiveness of the pursuit based on the
social zone of the target person. In our ISERA, the typical values are,

\begin{eqnarray}
D_P =\hspace{-1mm} \left\{
\begin{array}{cc}
1 & \textrm{if } R_P>1.5m\\
0.6 & \textrm{if } 1m<R_P<1.5m\\
0.2 & \textrm{if } 0.3m<R_P<1m\\
0& \textrm{if } R_P<0.3m\\
\end{array}
\right.
\end{eqnarray}

\section{Experimentation and Results}\label{experimentation-and-results}

The people detection, tracking, and pursuit algorithm has been tested
both indoor and outdoor. The 2D Lidar has \(180\) degree towards the
front of the robot, which gives \(720\) sampling points with \(0.25\)
degree separating each point. The observation of clusters for leg
detection is limited to be within \(5\) meters. Cluster size is limited
to be within \(0.07m - 0.2m\) and consist more than 4 sampling points.
Clusters would be classified as legs if the confidence exceed a cut off
threshold of \(0.5\). For the pursuit controller, the area in front of
the robot are divided into \(72\) bucket zones.

\begin{figure*}[!t]
\centering
\includegraphics[width=6.5in]{Following.pdf}
\caption{ISERA tracks and follows person A (wearing blue shirt), while person B (wearing black shirt) is passing in between: (a) Person B is about to pass through between person A and ISERA. (b) For a few moments ISERA did not have line of sight of person A, and the tracker will start to decay. (c) ISERA regain line of sight and continued following person A} \label{fig:blocked}
\vspace*{4pt}
\end{figure*}

\subsection{People Detection and
Tracking}\label{people-detection-and-tracking}

A single target tracking tests are coordinated with a person walking
repeatedly from one start point and an end point following a line marked
on the floor, while ISERA is staionary. In a cluttered environment legs
could be occluded by other people or obstacles. This condition is
emulated by adding more people walking around while the primary target
is still following the straight path. In addition, tracking of two
targets of interest were conducted. The paths are two straight lines
located with a roughly \(65cm\) distance to each other.

The results of single and multiple target tracking are depicted in
Fig.\ref{fig_static_result_one_people} and
Fig.\ref{fig_static_result_two_people}, respectively. As can be seen,
the tracking was executed robustly even in the middle of cluttered
objects and some spurious leg measurements. It is observed that during
occlusions, target can sometime deviate from the ground truth due to
temporary uncertainties in the filter. By keeping the confidence level
in each track, we can maintain which target to be preserved for the next
filter cycle.

\subsection{People Following}\label{people-following}

A series of experiments in a outdoor public area with obstacles to
verify that ISERA can follow a person. The target person to be followed
is selected by having the target person standing in front of ISERA.
ISERA's pursuit controller maintains \(30cm\) distance from the person.
Some disturbance is emulated by having people walking around and in
between ISERA and the target. The results shows ISERA consistently
manage to track and follow the target person that walks below ISERA's
maximum speed (limited to \(1m/s\) in our hardware). When traveling
between obstacles, ISERA also able to avoid going too close to the
obstacle, and successfully follows the person.

\section{Conclusion}\label{conclusion}

In this paper, ISERA service robot algorithms to follow a target person
have been presented. The person tracker associate the leg to a person
and tracks based on the confidence level that is given by the leg
identifier. The pursuit algorithm could regulate control between
maintaining the distance to target person and to avoid obstacle. Future
works should mitigate possible fail scenarios when the person walks too
fast or being obstructed for a prolong period of time.

\small

                                    \renewcommand\refname{References}
                        \bibliography{paperIROS2016}

    

\end{document}
