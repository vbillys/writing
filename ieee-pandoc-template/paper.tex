

\newcommand{\CLASSINPUTtoptextmargin}{104pt}
\newcommand{\CLASSINPUTbottomtextmargin}{54pt}
\newcommand{\CLASSINPUToutersidemargin}{37pt}
\newcommand{\CLASSINPUTinnersidemargin}{54pt}

%\renewcommand{\IEEEtitletopspaceextra}{18pt}
%\documentclass[10 pt,a4paper,conference]{IEEEtran}
%\documentclass[10 pt,a4paper,conference]{IEEEconf}
\documentclass[10 pt,a4paper,conference]{IEEEtran}


\usepackage{amssymb,amsmath}
\usepackage{ifxetex,ifluatex}
\usepackage{fixltx2e} % provides \textsubscript

\ifnum 0\ifxetex 1\fi\ifluatex 1\fi=0 % if pdftex
  \usepackage[T1]{fontenc}
  \usepackage[utf8]{inputenc}
\else % if luatex or xelatex
  \ifxetex
    \usepackage{mathspec}
    \usepackage{xltxtra,xunicode}
  \else
    \usepackage{fontspec}
  \fi
  \defaultfontfeatures{Mapping=tex-text,Scale=MatchLowercase}
  \newcommand{\euro}{€}


\fi
% use upquote if available, for straight quotes in verbatim environments
\IfFileExists{upquote.sty}{\usepackage{upquote}}{}
% use microtype if available
\IfFileExists{microtype.sty}{%
\usepackage{microtype}
\UseMicrotypeSet[protrusion]{basicmath} % disable protrusion for tt fonts
}{}


\ifxetex
  \usepackage[setpagesize=false, % page size defined by xetex
              unicode=false, % unicode breaks when used with xetex
              xetex]{hyperref}
\else
  \usepackage[unicode=true]{hyperref}
\fi
\hypersetup{breaklinks=true,
            bookmarks=true,
            pdfauthor={},
            pdftitle={Autonomous Service Vehicle Development for Urban Road Navigation},
            colorlinks=true,
            citecolor=blue,
            urlcolor=blue,
            linkcolor=magenta,
            pdfborder={0 0 0}}
\urlstyle{same}  % don't use monospace font for urls
\usepackage[numbers,square]{natbib}
\bibliographystyle{IEEEtranN}


\usepackage{listings}

\usepackage{algorithm}
\usepackage{algorithmic}
%\usepackage{algpseudocode}
%\usepackage[]{algorithm2e}
\usepackage{pifont}


%\makeatletter
%\algrenewcommand\ALG@beginalgorithmic{\footnotesize}
%\makeatother

\usepackage{tabularx}
\usepackage{array}






\usepackage{graphicx,grffile}
\makeatletter
\def\maxwidth{\ifdim\Gin@nat@width>\linewidth\linewidth\else\Gin@nat@width\fi}
\def\maxheight{\ifdim\Gin@nat@height>\textheight\textheight\else\Gin@nat@height\fi}
\makeatother
% Scale images if necessary, so that they will not overflow the page
% margins by default, and it is still possible to overwrite the defaults
% using explicit options in \includegraphics[width, height, ...]{}
\setkeys{Gin}{width=\maxwidth,height=\maxheight,keepaspectratio}
\graphicspath{ {./figures/} }



%\setlength{\parindent}{0pt}
\setlength{\parskip}{6pt plus 2pt minus 1pt}
\setlength{\emergencystretch}{3em}  % prevent overfull lines
\providecommand{\tightlist}{%
  \setlength{\itemsep}{0pt}\setlength{\parskip}{0pt}}
\setcounter{secnumdepth}{5}


\renewcommand{\IEEEtitletopspaceextra}{18pt}
%\title{Autonomous Service Vehicle Development for Urban Road Navigation}
\title{\LARGE \bf Autonomous Service Vehicle Development for Urban Road Navigation}

\author{
	\IEEEauthorblockN{
		Saputra, V. B.\IEEEauthorrefmark{1}, 
		Ravichandiran, B.\IEEEauthorrefmark{2}, 
		Kumar Limbu, D.\IEEEauthorrefmark{6}, 
		Wong, C. Y.\IEEEauthorrefmark{4} 
		}
	\IEEEauthorblockA{%
		Institute for Infocomm Research \\
		1 Fusionopolis Way, Singapore 138632 \\
		Email: (\IEEEauthorrefmark{1}saputravb  \IEEEauthorrefmark{2}ravib  \IEEEauthorrefmark{6}dklimbu  \IEEEauthorrefmark{4}cywong  ) @i2r.a-star.edu.sg
	}
}


\date{}

\usepackage{color}

\definecolor{lightred}{RGB}{175,50,50}
\definecolor{lightgreen}{RGB}{0,150,0}
\definecolor{lightblue}{RGB}{50,50,175}

\lstdefinelanguage{diff}{
  morecomment=[f][\color{lightblue}]{diff },
  morecomment=[f][\color{lightblue}]{index },
  morecomment=[f][\color{lightblue}]{@@},     % group identifier
  morecomment=[f][\color{lightred}]-,         % deleted lines
  morecomment=[f][\color{lightgreen}]+,       % added lines
  morecomment=[f][\color{lightblue}]{---}, % Diff header lines (must appear after +,-)
  morecomment=[f][\color{lightblue}]{+++},
}
\hyphenation{}

\newcommand{\attn}[1]{{\color{red}#1}}
\newcommand{\desc}[1]{{\emph{\color{blue}#1}}}
\newcommand{\needcite}{\attn{\tiny{[cite]}}}
\newcommand{\todo}[1]{\colorbox{yellow}{\bf TODO: #1}}


\lstset{ %
  basicstyle=\ttfamily\footnotesize,        % the size of the fonts that are used for the code
  breakatwhitespace=false,         % sets if automatic breaks should only happen at whitespace
  breaklines=true,                 % sets automatic line breaking
  columns=fixed,
  keepspaces=true,                 % keeps spaces in text, useful for keeping indentation of code (possibly needs columns=flexible)
  numbers=left,                    % where to put the line-numbers; possible values are (none, left, right)
  numbersep=10pt,                   % how far the line-numbers are from the code
  numberstyle=\tiny, % the style that is used for the line-numbers
  showspaces=false,                % show spaces everywhere adding particular underscores; it overrides 'showstringspaces'
  showstringspaces=false,          % underline spaces within strings only
  showtabs=false,                  % show tabs within strings adding particular underscores
  stepnumber=1,                    % the step between two line-numbers. If it's 1, each line will be numbered
  tabsize=2,                       % sets default tabsize to 2 spaces
}

\urlstyle{rm}

% Redefines (sub)paragraphs to behave more like sections
%\ifx\paragraph\undefined\else
%\let\oldparagraph\paragraph
%\renewcommand{\paragraph}[1]{\oldparagraph{#1}\mbox{}}
%\fi
%\ifx\subparagraph\undefined\else
%\let\oldsubparagraph\subparagraph
%\renewcommand{\subparagraph}[1]{\oldsubparagraph{#1}\mbox{}}
%\fi

\begin{document}
\maketitle

\begin{abstract}
As the sensors and actuators for building an autonomous vehicle are
getting increasingly affordable, both industry and research now could
leverage on autonomous vehicle platform for various purposes. This paper
presents a study of realizing a class of autonomous vehicle called
Autonomous Service Vehicle (ASV) in the context of urban road
navigation. The goal is to apply autonomous vehicle technology for human
service oriented tasks. The ASV design is made for general service
vehicle and it is intended to be reusable in term of hardware and
software infrastructure. Components of the ASV are presented and
experimentation was conducted to validate and evaluate the ASV. Further
directions and necessary customization are explained in order to adapt
the ASV design for various projects.
\end{abstract}

%%\begin{IEEEkeywords}
%    %automobiles;
%    %feature extraction;
%    %intelligent transportation systems;
%    %mobile robots;
%    %optical scanners;
%    %path planning;
%    %trajectory control;
%    %autonomous service vehicle;
%    %drive-by-wire system;
%    %driverless car;
%    %electronic car;
%    %lane detection algorithm;
%    %lane keeping assistant;
%    %laser scanner;
%    %map free lane following;
%    %path planning;
%    %vehicle trajectory;
%    %Laser radar;
%    %Mobile robots;
%    %Path planning;
%    %Roads;
%    %Testing;
%    %Vehicles;
%    %Wheels%\end{IEEEkeywords}
%


\section{Introduction}\label{introduction}

Recent advancements in robotics technology have brought great benefit to
autonomous vehicle research. Since the Defense Advanced Research
Projects Agency (DARPA) 2003 Grand Challenge
\citep{ventures2006stanley}, many research groups have focused their
effort to solving problems related to autonomous vehicle navigation. As
laboratories and research centers bringing the autonomous vehicle
technologies to real world applications \citep{Ford_Jan_16}, acceptance
of fully unmanned vehicles will still be a challenge. However, with
current progress it can be shown that for a specific problem domain, in
this case service vehicles, the autonomous vehicle can be utilized for
automation of human service oriented tasks, such as automated road
cleaning, grass cutting and leaves sucking machine. This is supported by
the growth of service robotics industry in the recent years
\citep{SVR_Nov_15}. \citeauthor{7225673} {[}@7225673{]} introduced
mobile service robot for electrical power plant inspection. This in turn
could reduce or replace human involvement in manual inspection process.
\citeauthor{6629559} {[}@6629559{]} presented an autonomous driving car
for research purposes that includes features that allows for urban road
navigation. Similarly, \citeauthor{7225765} {[}@7225765{]} introduced a
more advanced autonomous driving platform consisting of high performance
cameras, LIDARs, and other navigation sensors. While these platforms are
very attractive in term of their capability of handling big streams of
sensors data, the demanding computing power renders them unsuitable for
small service vehicle. The Autonomous Service Vehicle (ASV) proposed in
this paper is developed as a prototype to demonstrate autonomous vehicle
capability tasked for a service vehicle.

Autonomous platforms are usually retrofitted from manually driven
vehicles. This approach is commonly used because it is the most viable
and affordable way to install a Drive-By-Wire (DBW) system into existing
vehicle rather than to entirely build a new one. It was also studied
that automation driver's task has positive impact to improving driving
safety \citep{stanton1996fly}. Various effort to assist drivers has lead
to development of ADAS system {[}@6232125;@7225760{]}. A DBW enables
computerized control of the vehicle by means of replacing human drivers
with actuators, which are usually driven by electrical motors
{[}@1035218{]}. The ASV presented in this paper is retrofitted from a
compact electric vehicle \citep{SongZW_IV_2015}, and thus we studied the
feasibility of converting this class of vehicles into service vehicles.

This paper contribution is threefold. First, to describe and propose a
generic design of the system architecture based on necessary components
for the ASV applications. Second, to propose the ASV architecture to be
adopted into generic service platform for both research and application
purposes. Although, the implementation of the ASV presented here are
subject to certain sources, the components are generic and should be
reproducible with similar effort. Lastly, to present discussion on
issues surfaced during tests and evaluation of the ASV navigation in
urban road environment.

\section{Architecture}\label{architecture}

The main service types for ASV are related to defense, agriculture and
inspection, logistics and medical applications \citep{SVR_Nov_15}. As a
platform, ASV is required to navigate through environment to complete
the assigned service tasks. The approach taken is to modify current
vehicle to enable automated control by means of processing sensors data.
In this paper, the ASV components are designed to complete a case of
urban road navigation. Components of the proposed ASV can be summarized
in Fig. \ref{fig_1}.

\begin{figure}[!t]
\centering
\includegraphics[width=3.5in]{ecoms_arch.pdf}
\caption{Block Diagram of the ASV system.}
\label{fig_1}
\end{figure}

\subsection{Middleware}\label{middleware}

Middleware refers to the software component that connects the various
modules across network of distributed system. It makes communication
transparent to connected modules by providing reliability, consistency,
security, and privacy capabilities. In the ASV architecture CHROMOSOME
(abbreviated as XME) \citep{CHROMOSOME}, a message-oriented middleware
is used. XME implements the publish-subscribe message paradigm, which
provides guarantees with respect to the behaviour of distributed
applications \citep{buckl2014chromosome}. The other alternative
middlewares are Robotics Operating System (ROS) \citep{quigley2009ros}
and OpenRTM-aist \citep{ando2005rt} which are commonly used in the field
of robotics.

\subsection{Localization}\label{localization}

\begin{figure}[!t]
\centering
\includegraphics[width=2.5in]{localize_2.pdf}
\caption{Feature Detection and Particle Filter of Localization process.}
\label{fig_localize}
\end{figure}

As opposed to indoor localization, the environment where ASV usually
deployed is outdoor and GPS data is available. However, it is studied
that GPS data is often contain errors that is too large for autonomous
navigation {[}@6629559{]}. One approach to compensate the error is to
use local lane marker or road boundary information to adjust lateral
position of the vehicle with respect to the road \citep{loc_Boss}. This
allow the vehicle control to always offset the vehicle position to
always stay on the drivable area on the road. However, it can be argued
that the position in longitudinal direction along the road with this
approach is not accurate. For service applications, it is necessary that
both lateral and longitudinal position of the vehicle are determined
accurately as the ASV often need to stop at specific point in its route.
We developed a hybrid approach that fuses both information from GPS and
a prior map based online localization. The method is based on Particle
Filter or sequential Bayes filter that involves prediction from vehicle
motion model and measurement update cycle {[}@6550131{]}. The
measurement update computed from GPS and the feature scan matching. This
approach is suitable for service applications because the size of the
map is ranging from small to medium scale up to a few kilometer squares.
With the current computing technology, the map of this size can be fully
loaded into memory.

Before deployment, the ASV needs build its feature map by collecting
data during manual driving. A map is built by aggregating the data from
laser range sensors installed in the ASV. This is implemented in
two-staged process. In the first stage, raw scan data is collected and
sequentially aligned with an Iterative Closest Point (ICP) method
\citep{censi08plicp}. During alignment, raw point cloud map is built by
registered point cloud and any loop closure will improve consistency of
the detected vehicle positions during mapping \citep{tiar2015fast}. In
the second stage, features along lateral and longitudinal direction are
gathered and registered into processed feature map. The dominant
features used for urban environment are road curbs and tree trunks. The
result of feature detection, and the ASV Particle Filter map based
localization can be shown in Fig. \ref{fig_localize}.

\subsection{Obstacle Detection}\label{obstacle-detection}

During operation of the ASV, obstacle obstructing or near the vehicle
should be considered. Thus, by using a laser range sensor model for
detection, a set $L$ of $n$ points which each has a range $r_{i}$ and a
angle $\alpha_{i}$, measured from laser beams
$L=\left\{(r_{1},\alpha_{1}),...,(r_{n},\alpha_{n})\right\}$ are
clustered to extract objects \citep{Juric-Kavelj2008_407}. A Gaussian
smoothing is applied to remove spurious noise before processing. Two
subsequent points belong to one cluster satisfies the distance threshold
$d(r_{i},r_{i+1}) \leq D_{0}+D_{1}min(r_{i},r_{i+1})$, where $D_{0}$ is
the accuracy of the range data.
$D_{1} = \frac{3}{2}(\alpha_{i+1}-\alpha_{i})$ is a constant that
determine maximum change allowed. Point that break the threshold will be
the start of another cluster. Processed laser scanner is represented by
a set of clusters $\left\{ C_{1},...,C_{m} \right\}$. Each cluster
consists of tuples
$({}^{C}r_{i},{}^{C}\alpha_{i},{}^{C}x_{i},{}^{C}y_{i})$ where
${}^{C}x_{i}$, ${}^{C}y_{i}$ are the Cartesian coordinate
representations.

Each cluster is a potentially detected object. Based on the lateral
proximity of each cluster, object boundaries are found by determining if
each cluster satisfies $w_{min} \leq d(y_{min},y_{max})  \leq w_{max}$,
where $w_{max}$ and $w_{min}$ depend on widths of object of interest.
The detected obstacles are sent the path planner to aid in calculation
of the navigation course. Moreover, obstacle detection improves false
recognition removal of map objects for localization {[}@5164269{]}. The
cluster points of detected objects are used to determine areas of where
measured points should be excluded from the localization and mapping
process.

\subsection{Path Planning}\label{path-planning}

The Path planning module generate reference trajectory for the ASV to
follow from a start to a destination point. The ASV path planning is
implemented in two stages: Global Path Planning and Local Path Planning.

\subsubsection{Global Path Planning}\label{global-path-planning}

Global path planning aims to generate the waypoints to connect the
destination and the start point. In the literature, global path planning
can be achieved by the notable path planning algorithm such
$A^{*}$\citep{hart1968formal}, $D^{*}$\citep{stentz1994optimal},
Rapidly-Exploring Random Trees\citep{lavalle1998rapidly}, Probabilistic
roadmaps\citep{kavraki1996probabilistic}. For the ASV, a teach and
repeat solution is prefered as the planned path rarely change. This
implies that once the ASV is taught to navigate through certain
waypoints, it will be able to repeat it and more complex routing can be
achieved by combining multiple sets of waypoints. The ASV obtain its
trajectory by recording the vehicle position provided by localization
during manual driving. The trajectory obtained is further optimized by
post processing Alg. \ref{alg_global_waypoints}. In post processing
duplicate vehicle position points, points lie on straight line, curved
points within vehicle turning radius are removed to generate the
optimized path.

\begin{algorithm}
      \scriptsize
      \algsetup{linenosize=\scriptsize}
      \caption{Global Path Planning}
      \label{alg_global_waypoints}
      \begin{algorithmic}[1]
          \REQUIRE minLen, minRad, recordedPoint[] \COMMENT{array contains x, y and $\theta$}
            \STATE initialize points[], globalPoints[] \COMMENT{array contains x, y and $\theta$}

            \STATE totalGlobalPoints = 0
            
            \FORALL{$points$ in $recordedPoint$}
                \IF{$totalGlobalPoints$ == 0}
                    \STATE append $points$ to $globalPoints$
                    \STATE $totalGlobalPoints++$
                \ELSE
                    \IF{((distance between $points$  and $globalPoints[totalGlobalPoints - 1]$) $\geq$ $minLen$) $\AND$ (($\arctan$ between $points$ and $globalPoints[totalGlobalPoints - 1]$) $\geq$ minRad)}
                        \STATE append $points$ to $globalPoints$
                        \STATE $totalGlobalPoints++$
                    \ENDIF
                \ENDIF
            \ENDFOR
      \end{algorithmic}
\end{algorithm}

\subsubsection{Local Path Planning}\label{local-path-planning}

Local Path Planning aims to generate dynamic reference trajectories with
vehicle dynamics constraints such as turning radius and maximum
velocity. Based on Global Path Planning trajectory, Local Path Planning
is recursively generated during vehicle on motion with specified
interval to achieve smoother vehicle navigation. Local Path Planning
mainly helps to deal with dynamic circumstances, such as avoiding
obstacle along the global path and to perform smoother turning based on
velocity constraint. The Local Path Planning module is developed with
Dubins-Curve \citep{DubinsCurves} library which is based Dubins
\citep{dubinscurve1957} path method. Dubins path is one of the simplest
geometric methods to compute the shortest path between two points in the
Euclidean plane with turning radius as constraint. Dubins path is
suitable for nonholonomic wheeled mobile robot or car like autonomous
vehicles. Dubins path is built on top of the assumption that
non-holonomic vehicle has a typical motion of going straight (S),
turning left(L) and turning right(R). According to Dubins
\citep{dubinscurve1957} optimal path will be always one of the six
possible motion sequences: \{LSL, RSR, RSL, LSR, LRL, RLR\}. For details
on Dubins method, readers can refer to section 15.3.1 in
\citep{lavalle2006planning}. Local Path Planning Alg.
\ref{alg_local_waypoints} is executed at regular interval when the AVS
has travelled a certain distance or any obstacle is detected along the
path. In the event of obstacle obstructing the global path, the local
path will be shifted to the left or right side of obstacle boundary
based on environmental factors for safer avoiding. However, if the
obstacle is an moving object, the planner is unable to avoid. Thus, in
this case velocity profile will be lowered down based on a safe distance
threshold. An example of the Local Path Planning avoidance is shown on
Fig. \ref{veh_avoid_close}.

\begin{figure}[!t]
\centering
\includegraphics[width=2.0in]{vehicle_travelled_crop.pdf}
\caption{Vehicle avoidance path generated by local path planning}
\label{veh_avoid_close}
\end{figure}

\begin{algorithm}
      \scriptsize
     \algsetup{linenosize=\scriptsize}

      \caption{Local Path Planning}
      \label{alg_local_waypoints}
      \begin{algorithmic}[1]
          \REQUIRE currentPos, obstaclePoints[], globalPoints[], turnRadius, stepSize, maxLen
            \STATE initialize localPoints[], startPoint[], endPoint[] \COMMENT{array contains x, y and $\theta$}
            \STATE wayLen = 0
            
            \STATE $globalClose \leftarrow$ Find closest $globalPoints$ to $currentPos$ \COMMENT{closest points based on Euclidean distance}
            
            \STATE $startPoint$ = $globalPoints[globalClose]$           
            
            \WHILE{$startPoint$ is not a last globalPoint $\AND$ $wayLen \leq maxLen$ }
                
                \STATE $endPoint$ = $globalPoints[startPoint + 1]$
            
                \IF{$obstaclePoints$ exists between $startPoint and endPoint$}
                \STATE $endPoint \leftarrow$ will be shifted point where obstaclePoint exist. \COMMENT{Based on environment path shift will either left or right}
                \ENDIF                  
                
                \STATE $wayLen \leftarrow$ add Dubins Path Length for $startPoint$ and $endPoint$
                \STATE $localPoints \leftarrow$ append Dubins Path for $startPoint$ and $endPoint$ with given $turnRadius$ and $stepSize$
                \STATE $startPoint$ = $endPoint$
            \ENDWHILE
            
      \end{algorithmic}
\end{algorithm}

\subsection{Control}\label{control}

Vehicle control performs necessary action to follow the intended path
generated by path planner for smooth navigation towards destination
point. The gas and brake control for regulating vehicle velocity along
longitudinal direction will be handled by the velocity control module.
The steering control for turning vehicle along lateral direction will
handled by the lateral control module.

\subsubsection{Velocity control}\label{velocity-control}

Velocity control controls the vehicle velocity for along longitudinal
direction by controlling the gas, brake motor position in the drive by
wire system. Velocity control aims to achieve the desired velocity
determined by the module's input such as path planning, obstacle
detection and health monitoring report. For smoother control and to
avoid jerkiness of the vehicle, speed profiling is needed. Speed
profiling is the process to estimates the speed based on the curvature
of the path and speed limits for specific environments along vehicle
trajectory. The estimation is calculated from the aggregation of the
data collected during manual driving and maximum/minimum speed limits
are applied. During local path planning, speed profiling is computed to
determine the optimal speed.

To achieve the desired velocity, a closed loop
Proportional-Integral-Derivative (PID) controller is implemented to
compute the desired gas/brake position of the DBW motor. The closed loop
PID controller helps in noise rejection and robustness of control with
design flexibility. Based on desired velocity, the controller will
calculate the motor position and passes to drive by wire (DBW) system.
The wheel encoder measures the speed precisely and provide feedback to
the controller. The block diagram of the velocity control implementation
is shown in Fig \ref{fig_vel_control}. The governing control law of the
PID controller is given in the following:

\begin{equation}
\label{eqn_pid}
GB = k_p{E} + k_{i}\int_{0}^{t}{E}\,{dt} + k_{d}\frac{dE}{dt}
\end{equation}

Where $GB$ is gas-brake position value, $E$ is the velocity error, $dE$
is the differential error, $k_p$ is the proportional gain, $k_i$ is the
integral gain, $k_d$ is the differential gain.

\begin{figure}[!t]
\centering
\includegraphics[width=3.5in]{pid_velocity.pdf}
\caption{Velocity control based on closed loop PID.}
\label{fig_vel_control}
\end{figure}

\subsubsection{Lateral Control}\label{lateral-control}

Lateral control aims to control the steering angle of the vehicle based
on trajectory generated by path planner. The lateral controller is
developed based on Stanley's \citep{ventures2006stanley} steering
controller. Stanley steering controller is the geometric path tracking
approach used by Stanford university's autonomous vehicle during DARPA
grand challenge 2006. The closed-loop geometric tracking controller
helps vehicle to follow the desired path generated by local path
planning. The controller is based on non-linear cross track error
$C_{e}$ and heading error $\theta_{e}$. Cross track error is the lateral
distance between front axle center point $(P_{x}, P_{y})$ and closest
reference point $(R_{x}, R_{y})$ for which exponential can be shown
\citep{ventures2006stanley}. This term helps the vehicle to steer
towards the path to yield $C_{e}$ as zero when its following the path.
The heading error $\theta_{e}$ is calculated by difference in vehicle
heading $\theta$ and reference heading $\theta_{r}$ of path at
$(R_{x}, R_{y})$. This error term directly applies to steering angle
$\delta$ with some steering angle limitation.

\begin{equation}
\label{eqn_heading_error}
\theta_{e} = \theta - \theta_{r}
\end{equation}

When the cross track error $C_{e}$ is non-zero, it adjusts $\delta$ such
that the intended trajectory intersects the reference path tangent from
$(R_{x}, R_{y})$ at velocity in time $V_{t}$ units from the front axle
center point $(P_{x}, P_{y})$. The resulting steering control law by
combining the two terms are shown in (\ref{eqn_stanley_equation}).
$k_{h}$ and $k_{c}$ are the gain constants for heading error and cross
error, respectively. Fig. \ref{fig_stanley} illustrates the geometric
relationship of control parameters.

\begin{equation}
\label{eqn_stanley_equation}
\delta(t) = k_{h}\theta_{e} + tan^{^{-1}}\left (\frac{k_{c}C_{e}(t)}{V_{x}(t)}  \right )
\end{equation}

\begin{figure}[!t]
\centering
\includegraphics[width=1.5in]{stanley_2.pdf}
\caption{Stanley controller geometry}
\label{fig_stanley}
\end{figure}

\subsection{Health Monitoring}\label{health-monitoring}

Safety is very important in a critical mission project like autonomous
service vehicle. Not only that robust operation is required, but also
safety must be prioritized in the event of malfunctions or catastrophic
failures. Health monitoring module is a component in the ASV that acts
as a safety agent to react to those events in timely manner. It is based
on publish-subscribe model. This allows both centralized and distributed
monitoring model. In the initialization stage, each ASV functional
module component subscribes to a health topic proceeding completion of
self-diagnosis. The health topic is provided by the Central Health
Monitor (CHM). In the operational stage, modules are to send unique
description and event identification with timestamp to the CHM. The CHM
is responsible for reporting any important event to the user, log the
event, and take suitable action based on severity of the detected
failure or error. In addition, each module has their own reporting
module, but limited only to each module scope of operation. The severity
levels and the corresponding actions can be summarized in Table
\ref{table_risk_hm}.

\begin{table}[!t]
\renewcommand{\arraystretch}{1.3}
\caption{Severity Level in Health Monitoring}
\label{table_risk_hm}
\centering
\footnotesize
\begin{tabular}{p{1.5cm}||p{6.cm}}
  \hline
    \textbf{Severity} &
    \textbf{Actions} \cr
  \hline\hline
    None & Normal Operation. Status Logging. \cr
  \hline
    Warn & Display warning on user interface. Status logging. \cr
  \hline
    Abort & Vehicle aborts mission, plans and stops at a safe zone. Status logging and displaying on user interface. \cr
  \hline
    Emergency Stop & Vehicle aborts mission immediately. Status logging and displaying on user interface. All modules restart and perform self-diagnosis. \cr
  \hline
\end{tabular}
\end{table}

\subsection{Human Machine Interface}\label{human-machine-interface}

Human Machine Interface (HMI) is a graphical user interface where users
could control and monitor the ASV. We have developed a web based HMI, so
that the ASV could be monitored and controlled from any remote location.
Remote control functions include emergency stop immediate request, task
scheduling and change or reroute the ASV global path plan. The HMI is
web based and developed with websocket \citep{fette2011websocket}
protocol implementation for achieving low-latency, long-running
bidirectional communication. For security concern, HMI can be accessed
only from intranet with unique private IP assigned for the ASV and user
need to authenticate with username and password. Any HTML5 supported
browser can be used to access the HMI. The HMI consist of dashboard,
system interface and settings interface. They individually has different
scope of information and options. For example, Fig. \ref{fig_hmi} shows
the HMI dashboard. The dashboard is essential for vehicle monitoring
information such as precise ASV position on the map, vehicle speed,
battery level and system log for displaying critical error logs and
network status. The system interface is used to manage individual
modules. It has options to starting, to stop and to view error logs. The
settings interface is used to modify ASV parameters such as maximum
speed, braking distance, velocity control and lateral control gain.

\begin{figure}[!t]
\centering
\includegraphics[width=3.0in]{hmi.pdf}
\caption{HMI dashboard interface}
\label{fig_hmi}
\end{figure}

\section{Experimentation}\label{experimentation}

In our previous work \citep{SongZW_IV_2015}, we retrofitted the first
version of a compact electric vehicle. In this paper, we upgraded the
ASV system and improve installation by moving into the second version of
the vehicle. The second version vehicle provides better stability in
term of maneuverability, and provide more space at the back of passenger
seat to install complete power supply system and necessary computing
devices. The current ASV prototype is shown in Fig. \ref{coms1_bare}.

\begin{figure}[!t]
\centering
\includegraphics[height=1.7in]{coms2_1.pdf}
\caption{Vehicle platform for the ASV.}
\label{coms1_bare}
\end{figure}

\subsection{Setup}\label{setup}

Placement of the components on the autonomous vehicle can be seen in
Fig. \ref{ipe_ex}, and their connection to the system can be summarized
in Fig. \ref{dia_ex}. As can be seen, the vehicle has a dedicated box
originally tailored for carrying load. This box is revamped so that it
has more space to contain all the components. The ASV system is
integrated through a Central Processing Unit (CPU) and a Sensor Actuator
Hub (SAH). Both CPU and SAH are compact PCs, but for different purpose.
All the modules including HMI and CHM are run on the CPU. SAH gathers
data from sensors and from the Vehicle Interface (VI). On top of that,
it also bridges the CPU with the DBW control and emergency stop signal
can bypass control to instruct immediate brake. In order to provide
real-time logging and a database server system is installed in a low-end
compact PC. This log database server receives data from all modules, and
writes to storage in timely manner so that in the event of sudden power
loss or vehicle breakdown, most recent data can be recovered and
analyzed.

\begin{figure}[!t]
\centering
\includegraphics[height=1.6in]{ipe_ex.pdf}
\caption{Placement of components in the ASV.}
\label{ipe_ex}
\end{figure}

\begin{figure}[!t]
\centering
\includegraphics[width=1.9in]{dia_ex.pdf}
\caption{Diagram of components connection.}
\label{dia_ex}
\end{figure}

\subsubsection{Drive-By-Wire}\label{drive-by-wire}

\begin{figure}[!t]
\centering
\includegraphics[width=2.5in]{Overview.jpg}
\caption{Drive-by-Wire installation of the ASV.}
\label{fig_DBW}
\end{figure}

The original vehicle comes with steering and gas/brake pedals. The
Drive-By-Wire (DBW) refers to the mechanics added on to the vehicle so
that it can be controlled by computer. Two motors are installed, each
for controlling the steering and the pedals. The steering actuation is
achieved by modifying the steering column and combining it with the
steering motor shaft. The pedals actuation is designed so that only one
pedal is pressed at one time. This actuation is achieved by installing
two levers on top of each pedal, and connecting them to the second
motor: one lever via direct linkage, and the other lever via a metal
wire in such a way that individual pedal can be pressed by selecting
motor rotating direction. This removes the need for a third motor, and
also increases safety that it is mechanically impossible to press both
pedals in case there is error in the motor controller. The electric
vehicle only has two gear options linked directly to the internal motor
inverter as it does not have mechanical gear, but the gear is needed to
determine the direction of travel. Thus, the gear selection can be
controlled using electrical signal through the VI.

\subsubsection{Power Distribution}\label{power-distribution}

\begin{figure}[!t]
\centering
\includegraphics[width=2.2in]{power_diagram.pdf}
\caption{Electrical diagram of the ASV.}
\label{fig_power}
\end{figure}

The power supply for the CPU, DBW, sensors and other peripherals is
provided by a separated rechargeable Lithium Polymer (LiPo) batteries.
The autonomous operation runs independently on the LiPo batteries to
avoid complication that may arise if the main batteries is tapped.
Furthermore, power supply for ASV is crucial as the autonomous system
depends on it. Therefore, additional Power Supply Units (PSU) and a UPS
are installed such as shown in Fig. \ref{fig_power}. There are multiple
redundancies power line to secure fail-safe and fault-tolerant operation
of the ASV during its navigation. In addition, VI unit is made of
multiple stack of microcontroller and relay boards in order to ensure
safe interfacing to/from the vehicle. Fans are installed to regulate
temperature inside the box.

\subsubsection{Sensors}\label{sensors}

Autonomous navigation of the ASV heavily depends on the perception
system which data comes from various sensors.

\paragraph{Odometry}\label{odometry}

Odometry position is estimated via accumulating the motion sensors data
over time. Pure odometry position is not recommended for long-term
navigation, since the motion sensor error will accumulate drift in the
actual position over time. The navigation relies on the localization
module to fuse odometry data with GPS data and the developed map based
localization. The odometry position is derived from the
(\ref{eqn_simple_odometry}). The following two motion sensors are
installed in the ASV: a wheel motion sensor (Kistler Wheel Pulse
Transducers) to calculate vehicle travelled distance, and GPS/IMU sensor
(Xsens MTi-G-700 GPS) to measure global positions and heading rate.

\begin{equation}
\begin{split}
\label{eqn_simple_odometry}
\theta_t = \theta_{t-1} + \Delta\theta_{t} \\
x_t = x_{t-1} + (\Delta v_t + cos(\theta_t)) \\
y_t = y_{t-1} + (\Delta v_t + sin(\theta_t))
\end{split}
\end{equation}

where $\Delta\theta$ is the yaw rate obtained from IMU, $\theta$ is the
heading, $\Delta v$ is distance obtained from wheel motion sensor, $x$
and $y$ are the Cartesian coordinates of the odometry position, and $t$
is the time index which is increased every sampling time.

\paragraph{Laser Range Sensors}\label{laser-range-sensors}

LIDAR ( Light Detection and Ranging) is one of the popular light range
sensors used in various field of Robotics, Autonomous Vehicles, and
Geo-mapping, forestry, etc. LIDAR will provide the distance of the
object by emitting laser towards it and analysing its reflection.
Compared to stereo vision systems {[}@6856563{]}, LIDARs generally has
more field of view, longer detection range, and more accurate depth
information. A four layers Lidar (Ibeo ScaLa) is mounted on front bumper
facing forward for obstacle detection up to 100 meters. The obstacle
detection can be done in one layer, but the other layers will provide
redundancies if the ASV tilt forward or backward. In this case, one or
more layers can hit the ground or the free space below or above an
object, respectively. At least one layer will detect an object if there
is one. Another 16 layers Lidar (Velodyne Puck) which can sense
surrounding object and environment shape in 25m radius is used mainly
for localization. This LIDAR is mounted roughly $30^{\circ}$ towards
ground purposely for detecting main features of the environment such as
the road curbs and the tree trunks which are used for localization.

\paragraph{Camera}\label{camera}

The camera (Logitech C930e) is used for video data logging and remote
monitoring via the HMI interface. Camera video frames are read using
\emph{video4linux} API, which later encoded and streamed in \emph{WebM}
format with \emph{ffmpeg} and \emph{ffserver} setup. We did try to use
image processing to detect lane marker for navigation, but unfortunately
due to shadows and significant change in ambient light level during the
day, detection results were not stable. This issue can be solved by
replacing the camera with HDR camera or fuse the results with the LIDAR
point cloud intensity data.

\subsubsection{Safety System}\label{safety-system}

Safety is a primary concern for any autonomous system, AVS should be
stopped safely in any unexpected circumstances, such as sensor failures
or errors in software modules. In the AVS experiment, we have
implemented safety in three levels. First level is implemented in the
Health Monitoring system which monitors hardware, software modules and
takes necessary action on failures reported. The on-board emergency
switch which provides the second level of safety, can be accessed while
there is a person inside the ASV. Third level is implemented by a remote
software emergency stop through the HMI interface and a hard-wired radio
emergency stop system which works on 900MHz and 2.4GHz frequency.

\subsection{Testing Results and
Evaluation}\label{testing-results-and-evaluation}

Testing is carried out in an urban environment for the distance of 1km
range. As explained in the global path planning, waypoints record by
manual driving and processed. Figure \ref{map_waypoint} shows the map of
the urban environment road structure and global trajectory for testing.
For autonomous driving, the maximum velocity is capped by 4m/s and
static obstacles are placed in 3 locations as shown on Fig.
\ref{veh_path_overview}. Figure \ref{vel_error_test} shows the output of
velocity control with respective to desired velocity. Velocity
controller achieves desired velocity linearly and little noise is also
observed during constant velocity. Between 900 to 950 cycle time there
is a sudden drop of velocity is noticed, which is caused by dynamic
obstacle along the path that caused the vehicle to slow down. However,
the ASV velocity controller can recover speed to the desired profile
after the object is no longer detected.

Figure \ref{control_error_test} shows the output of lateral control such
as angle error, lateral distance error (cross-track error) and final
steering output. The control parameter of the ASV is tuned to tolerate
localization error up to 20 cm, thus the final steering output follows
angle error closely unless there is a steady-state lateral distance
error. This can be seen that eventhough the localization emits episodic
high-frequency noisy data due to bad perception or GPS signal, the
controller is able to filter the steering output. This results in smooth
driving experience if someone is inside the ASV. In the event of turning
in a higher error is observed. This is due to maximum steering
constraint which is set to 0.41 radians. Errors during turning is
expected to turn the steering to the direction of the path.

\begin{figure}[!t]
\centering
\includegraphics[width=2.5in]{map_overlay.pdf}
\caption{Recorded Waypoint to travel}
\label{map_waypoint}
\end{figure}

\begin{figure}[!t]
\centering
\includegraphics[width=2.5in]{vehicle_travelled_label.pdf}
\caption{Vehicle path overview}
\label{veh_path_overview}
\end{figure}

\begin{figure}[!t]
\centering
\includegraphics[width=2.5in]{velocity_error_1.pdf}
\caption{Velocity Control output during testing }
\label{vel_error_test}
\end{figure}

\begin{figure}[!t]
\centering
\includegraphics[width=2.5in]{control_error_1.pdf}
\caption{Lateral Control output during testing }
\label{control_error_test}
\end{figure}

\subsection{Adaptation for specific
service}\label{adaptation-for-specific-service}

Depending on the specific task, various tools which include sensors
and/or actuators that are necessary for the service task can be mounted
onto the ASV. These sensors and actuators are collectively called as a
workload. The ASV modules are customizable to carry the workload and
perform the desired service task. The ASV localize itself by using
information from map and GPS, thus eliminating the need for installing
environmental guidance such as RFID {[}@7225673{]} or magnetic rails
{[}@5354591{]}. For inspection and surveillance purposes, the camera
data can be streamed and recorded via the HMI. For cleaning service, the
ASV perform its task during its movement. This necessitates that the
path planning follows certain objects, such as road pavement or curb.
Combined with the method proposed in \citep{SongZW_IV_2015}, the ASV can
navigate to track arbitrary distinguishable object boundaries to conduct
the cleaning. Nevertheless, thanks to the developed localization,
arbitrary path rerouting is also possible. For services that require
following certain moving object such as pedestrian {[}@7139259{]}, the
LIDAR and camera sensors could be used to track the object.
Subsequently, the tracked object positions will become the destination
points for the ASV path planner.

\section{Conclusion}\label{conclusion}

The ASV has been developed to handle human service tasks. It has generic
architecture which is reproducible for other purpose of use. During the
development of the ASV, issues had surfaced, and dealt with current
approaches. The main function of the ASV is to navigate through set of
prior taught waypoints. Precise localization is achieved by combining
GPS reading and a prior map generated using LIDAR. This allows
adaptation to various human oriented service tasks, with appropriate
tooling mounted. The experimental prototype results verified that the
ASV is able to execute the path reliably and safely. Currently, the low
level controller has been implemented in an embedded microcontroller
platform. It is of current ongoing work that other functional modules
will be implemented in a compact, low cost and low power compute device,
enabling possibilities of efficiently retrofitting compact vehicles or
existing service machines into ASVs. This full embedded implementation
and more experiments on study to deal with specific services will be
conducted and reported in the future.

\small

                                    \renewcommand\refname{References}
                        \bibliography{papers}

    

\end{document}
