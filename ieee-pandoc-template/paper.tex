%\documentclass[10 pt,letterpaper,conference]{IEEEtran}
%\documentclass[10 pt,letterpaper,conference]{IEEEconf}
\documentclass[10 pt,letterpaper,conference]{IEEEtran}



\usepackage{amssymb,amsmath}
\usepackage{ifxetex,ifluatex}
\usepackage{fixltx2e} % provides \textsubscript

\ifnum 0\ifxetex 1\fi\ifluatex 1\fi=0 % if pdftex
  \usepackage[T1]{fontenc}
  \usepackage[utf8]{inputenc}
\else % if luatex or xelatex
  \ifxetex
    \usepackage{mathspec}
    \usepackage{xltxtra,xunicode}
  \else
    \usepackage{fontspec}
  \fi
  \defaultfontfeatures{Mapping=tex-text,Scale=MatchLowercase}
  \newcommand{\euro}{€}


\fi
% use upquote if available, for straight quotes in verbatim environments
\IfFileExists{upquote.sty}{\usepackage{upquote}}{}
% use microtype if available
\IfFileExists{microtype.sty}{%
\usepackage{microtype}
\UseMicrotypeSet[protrusion]{basicmath} % disable protrusion for tt fonts
}{}


\ifxetex
  \usepackage[setpagesize=false, % page size defined by xetex
              unicode=false, % unicode breaks when used with xetex
              xetex]{hyperref}
\else
  \usepackage[unicode=true]{hyperref}
\fi
\hypersetup{breaklinks=true,
            bookmarks=true,
            pdfauthor={},
            pdftitle={System Architecture for a Class of Small Service Vehicle},
            colorlinks=true,
            citecolor=blue,
            urlcolor=blue,
            linkcolor=magenta,
            pdfborder={0 0 0}}
\urlstyle{same}  % don't use monospace font for urls
\usepackage[numbers,square]{natbib}
\bibliographystyle{IEEEtranN-nourl}


\usepackage{listings}





\usepackage{graphicx,grffile}
\makeatletter
\def\maxwidth{\ifdim\Gin@nat@width>\linewidth\linewidth\else\Gin@nat@width\fi}
\def\maxheight{\ifdim\Gin@nat@height>\textheight\textheight\else\Gin@nat@height\fi}
\makeatother
% Scale images if necessary, so that they will not overflow the page
% margins by default, and it is still possible to overwrite the defaults
% using explicit options in \includegraphics[width, height, ...]{}
\setkeys{Gin}{width=\maxwidth,height=\maxheight,keepaspectratio}
\graphicspath{ {./figures/} }



%\setlength{\parindent}{0pt}
\setlength{\parskip}{6pt plus 2pt minus 1pt}
\setlength{\emergencystretch}{3em}  % prevent overfull lines
\providecommand{\tightlist}{%
  \setlength{\itemsep}{0pt}\setlength{\parskip}{0pt}}
\setcounter{secnumdepth}{5}


\title{System Architecture for a Class of Small Service Vehicle}

\author{
		\IEEEauthorblockN{Saputra, V. B.}
	\IEEEauthorblockA{%
		Institute for Infocomm Research \\
		1 Fusionopolis Way, Singapore 138632 \\
	saputravb@i2r.a-star.edu.sg}
	 \and
		\IEEEauthorblockN{Ravichandiran, B.}
	\IEEEauthorblockA{%
		Institute for Infocomm Research \\
		1 Fusionopolis Way, Singapore 138632 \\
	ravib@i2r.a-star.edu.sg}
	}

\date{}

\usepackage{color}

\definecolor{lightred}{RGB}{175,50,50}
\definecolor{lightgreen}{RGB}{0,150,0}
\definecolor{lightblue}{RGB}{50,50,175}

\lstdefinelanguage{diff}{
  morecomment=[f][\color{lightblue}]{diff },
  morecomment=[f][\color{lightblue}]{index },
  morecomment=[f][\color{lightblue}]{@@},     % group identifier
  morecomment=[f][\color{lightred}]-,         % deleted lines
  morecomment=[f][\color{lightgreen}]+,       % added lines
  morecomment=[f][\color{lightblue}]{---}, % Diff header lines (must appear after +,-)
  morecomment=[f][\color{lightblue}]{+++},
}
\hyphenation{}

\newcommand{\attn}[1]{{\color{red}#1}}
\newcommand{\desc}[1]{{\emph{\color{blue}#1}}}
\newcommand{\needcite}{\attn{\tiny{[cite]}}}
\newcommand{\todo}[1]{\colorbox{yellow}{\bf TODO: #1}}


\lstset{ %
  basicstyle=\ttfamily\footnotesize,        % the size of the fonts that are used for the code
  breakatwhitespace=false,         % sets if automatic breaks should only happen at whitespace
  breaklines=true,                 % sets automatic line breaking
  columns=fixed,
  keepspaces=true,                 % keeps spaces in text, useful for keeping indentation of code (possibly needs columns=flexible)
  numbers=left,                    % where to put the line-numbers; possible values are (none, left, right)
  numbersep=10pt,                   % how far the line-numbers are from the code
  numberstyle=\tiny, % the style that is used for the line-numbers
  showspaces=false,                % show spaces everywhere adding particular underscores; it overrides 'showstringspaces'
  showstringspaces=false,          % underline spaces within strings only
  showtabs=false,                  % show tabs within strings adding particular underscores
  stepnumber=1,                    % the step between two line-numbers. If it's 1, each line will be numbered
  tabsize=2,                       % sets default tabsize to 2 spaces
}

\urlstyle{rm}

% Redefines (sub)paragraphs to behave more like sections
%\ifx\paragraph\undefined\else
%\let\oldparagraph\paragraph
%\renewcommand{\paragraph}[1]{\oldparagraph{#1}\mbox{}}
%\fi
%\ifx\subparagraph\undefined\else
%\let\oldsubparagraph\subparagraph
%\renewcommand{\subparagraph}[1]{\oldsubparagraph{#1}\mbox{}}
%\fi

\begin{document}
\maketitle

\begin{abstract}
By the dawn of the new millennium, robotics has undergone a major
transformation scope and dimensions. This expansion has been brought
about by the maturity of field and the advances in its related
technologies. From a largely dominant focus, robotics has been rapidly
expanding into the challenges of the world. The recent technology
development demonstrated how the fast growing progress toward the of new
perception, control, and motion planning techniques allow autonomous
vehicles not only to travel significant distances in off-road, but also
to operate in urban scenarios. The expected impact in the commercial
sector for automotive manufacturers is equally if not more: autonomous
sensing and control constitute key technologies to vehicles the future
that might help save thousands of lives now lost in traffic accidents.
Imagine a large field of vehicles without drivers traversing 60 miles in
live, operating entirely without human guidance. A complex course
including, traffic circles, and parking lots, defined by just kilobytes
of data, several meters wide traveling down lanes only slightly wider,
using systems with an accuracy of several meters. Humans in full-size
unmanned vehicles at speeds up to 60 miles per hour. This actually would
become reality in the near future. This project aims to bring the recent
development on autonomous driving technology into a more practical and
marketable platform so that it benefits can really be advantageous to
real users.
\end{abstract}

%%\begin{IEEEkeywords}
%    %automobiles;
%    %feature extraction;
%    %intelligent transportation systems;
%    %mobile robots;
%    %optical scanners;
%    %path planning;
%    %trajectory control;
%    %autonomous service vehicle;
%    %drive-by-wire system;
%    %driverless car;
%    %electronic car;
%    %lane detection algorithm;
%    %lane keeping assistant;
%    %laser scanner;
%    %map free lane following;
%    %path planning;
%    %vehicle trajectory;
%    %Laser radar;
%    %Mobile robots;
%    %Path planning;
%    %Roads;
%    %Testing;
%    %Vehicles;
%    %Wheels%\end{IEEEkeywords}
%


\thanks{*This work was supported by Autonomous Vehicle Department, Institute for Infocomm Research}

\section{Introduction}\label{introduction}

The computational requirements for autonomous driving are very
demanding, because autonomous vehicles have to process large amount of
online and offline data in order to sense and understand their
environment within long range safety distances with high precision, make
and perform driving decisions at real-time. Most current autonomous
vehicle prototypes around the world employ powerful distributed systems
consisting of multiple heterogeneous processors and systems for sensory
data processing, sensor fusion, intelligent behavior, and driving
control. These types of prototype systems are bulky, costly and power
hungry, thus are not suitable for real world deployment. Recently, Audi
developed an all-in-one embedded platform code named zFAS which
incorporates multiple heterogeneous processors (ECU/MCUs, Application
Processors, GPUs, FPGA and a video detection subsystem) on a single
tightly integrated board with specialized communication architecture
(including Deterministic Ethernet) to suit autonomous driving needs. The
computing power equals to the entire electronics architecture of today's
Audi vehicles, yet at a fraction of the size, cost and power
consumption. Such a platform is invaluable to eventually bring
autonomous driving technologies to the mass market.

This paper aims to describe and propose a generic design of the system
architecture based on the chosen compute components and the
characteristics / requirements of sensor fusion and autonomous driving
applications, integration, test and characterization for a class of
small service vehicle.

\section{Architecture}\label{architecture}

Components of a typical autonomous or assistive driving system could be
summarized in Fig. \ref{fig_1}. The system heavily depends on various
types of sensory data and sensors of the same type are synced with one
another through data sync modules. To make sense of sensor data, raw
data processing is required for different types of available data.
Overlapping field of views from different sensors could be combined and
the processing result is fused through a Low Level Fusion module. This
module also does some filtering to eliminate local noises and thus, make
the information more robust to small disturbances. The output objects
and localization data from low level fusion modules are then sent to a
High Level Fusion Module whereby all objects and filtered sensor
information are fused together to form a world model.

\begin{figure}[!t]
\centering
\includegraphics[width=2.5in]{HighLevel.pdf}
\caption{Block Diagram of an autonomous vehicle driving system.}
\label{fig_1}
\end{figure}

\subsection{Middleware}\label{middleware}

Middleware refers to the software component that connects the various
modules across network distributed system. It makes communication
transparent to connected modules by providing reliability, consistency,
security, and privacy capabilities. In our architecture
CHROMOSOME\citep{CHROMOSOME} a Message-oriented middleware is used.
CHROMOSOME implements publish-subscribe message paradigm, which also
provides more guarantees with respect to the behaviour of distributed
applications. \citep{buckl2014chromosome} The other alternative
middlewares are Robotics Operating System (ROS) \citep{quigley2009ros}
and OpenRTM-aist \citep{ando2005rt} which are commonly used in the field
of Robotics.

\section{Components}\label{components}

\subsection{Drive-By-Wire}\label{drive-by-wire}

\subsection{Control}\label{control}

Fig \ref{fig_2} shows a control error during simulation. A Stanley
controller can be used for different type of course, such as parking,
low speed, and medium travels.

\begin{figure}[!t]
\centering
\includegraphics[width=2.5in]{fig_ex.pdf}
\caption{Example of matplotlib graph (vector graphic in the paper).}
\label{fig_2}
\end{figure}

\subsection{Path Planning}\label{path-planning}

Path planning done by \citeauthor{SongZW_IV_2015} has revealed the fact
that small service vehicle can be robustly navigate the path as long as
the localization gives its correct position. For more examples, readers
are encouraged to read \citep{Corley-etal_2011}. In addition,
\citet{Corley-etal_2012} confirmed that localization can be affected by
surrounding dynamic environment changes such as parked cars, and moving
pedestrians. \citet{Aalst-etal_2004} proposed a solution to this, but
later the work has been verified not working in
\citep{Abadi-etal_2008, Abebe-etal_2009}. For more references on this
problem, suggested readings are
\citep{Ackerman-Halverson_1998, Agrawal-etal_1998, Ali-etal_2012, Alipour-etal_2013}.

\subsection{Mapping}\label{mapping}

Procedure of mapping includes:

\begin{enumerate}
\def\labelenumi{\arabic{enumi}.}
\tightlist
\item
  Point cloud assembly
\item
  Point cloud registration
\item
  Extracting useful information
\end{enumerate}

\subsubsection{Point cloud assembly}\label{point-cloud-assembly}

The assembly process is aggregating several point cloud frames from
sensor scans with some process of point cloud matching and registration.

\subsubsection{Point cloud registration}\label{point-cloud-registration}

\subsubsection{Extracting useful
information}\label{extracting-useful-information}

\begin{itemize}
\tightlist
\item
  Road curb
\item
  Traffic signs/lights
\item
  Building features
\end{itemize}

\paragraph{Road curb}\label{road-curb}

Road curb is the most reliable feature in road boundary detection. It is
easy to detect and currently it is the baseline for localization.

\paragraph{Traffic signs/lights}\label{traffic-signslights}

\paragraph{Building features}\label{building-features}

\subsection{Localization}\label{localization}

The localization is done by modeling the problem using \emph{Gamma
function} satisfying \(\Gamma(n) = (n-1)!\quad\forall n\in\mathbb N\) is
via the \textbf{euler} integral:

\begin{equation}
\label{eqn_example1}
x = \sum\limits_{i=0}^{z} 2^{i}Q
\end{equation}

\begin{equation}
\label{eqn_example2}
\Gamma(z) = \int_0^\infty t^{z-1}e^{-t}dt\,.
\end{equation}

\begin{equation}
\label{eqn_example3}
\mathbf{V}_1 \times \mathbf{V}_2 =  \begin{vmatrix}
\mathbf{i} & \mathbf{j} & \mathbf{k} \\
\frac{\partial X}{\partial u} &  \frac{\partial Y}{\partial u} & 0 \\
\frac{\partial X}{\partial v} & \frac{\partial Y}{\partial v} & 0
\end{vmatrix}
\end{equation}

Equation (\ref{eqn_example2}) is convoluted by means of \emph{Gaussian
Distribution}. In other words, (\ref{eqn_example1}) depicts the spectrum
of (\ref{eqn_example2}). The \textbf{eigen }matrix for the point cloud
distribution is given in (\ref{eqn_example2}).

..Or, using numbering and some examples of referencing them given above.

\subsection{Path Planning}\label{path-planning-1}

Citations examples\ldots{}

Path planning done by \citeauthor{SongZW_IV_2015} has revealed the fact
that small service vehicle can be robustly navigate the path as long as
the localization gives its correct position. For more examples, readers
are encouraged to read \citep{Corley-etal_2011}. In addition,
\citet{Corley-etal_2012} confirmed that localization can be affected by
surrounding dynamic environment changes such as parked cars, and moving
pedestrians. \citet{Aalst-etal_2004} proposed a solution to this, but
later the work has been verified not working in
\citep{Abadi-etal_2008, Abebe-etal_2009}. For more references on this
problem, suggested readings are
\citep{Ackerman-Halverson_1998, Agrawal-etal_1998, Ali-etal_2012, Alipour-etal_2013}.

\subsection{Control}\label{control-1}

Vehicle control or autonomous navigation is one of the key component for
autonomous vehicle. Vehicle control performs necessary action to follow
the intended path generated by path planner for smoother navigation
towards destination point. The gas, brake control for longitudinal
direction will be handled by velocity control and the steering control
for turning across lateral direction will handled by lateral control
module.

\subsubsection{Velocity control}\label{velocity-control}

Velocity control will estimate the gas, brake motor position in the
drive by wire system for maintaining the desired velocity and braking
for the vehicle. The desired velocity of the vehicle is determined with
the modules input such as path planning, obstacle detection and health
monitoring system. For autonomous lateral control the fixed constant
velocity will not be a wise choice, since the navigation path is not
always straight line of travel. The path may consists of various turns
and unexpected circumstances like obstacles. For smoother path following
and avoid jerkiness of vehicle, speed profiling have to done based on
path curvatures and speed limits. Based on the data collected by manual
driving, the speed is calculated for various path curvatures (turns).
During the local path generation, the curvature of path is calculated
and desired speed will be assigned to each curvatures across the path.
To avoid rapid speed change and to make velocity transition smooth from
straight line to curvature areas velocity averaging is performed in the
direction of end to path start. Apart from speed determined by speed
profiler across the path, in event of obstacle detection vehicle will
also perform action like slow down or stops based on obstacle detected
distance. In the event of certain module error or network communication
failure, health monitoring system will perform stop action based on
severity of error. To achieve desired velocity, the controller need to
calculate the desired gas/brake position of the motor. Thus one of the
frequently used controller technique closed loop PID controller is
implemented. The closed loop PID controller helps in noise rejection and
robustness of control with design flexibility. Based on desired
velocity, the controller will calculate the motor position and passes to
drive by wire (DBW) system. Drive by wire system will convert controller
command to control signal and passes to the motor. The wheel encoder
will help to measure the speed precisely and feedback to the controller
for the necessary action. The block diagram of the velocity control
implementation is shown in Fig \ref{fig_vel_control}.

\begin{figure}[!t]
\centering
\includegraphics[width=2.5in]{pid_velocity.pdf}
\caption{Velocity control based on closed loop PID.}
\label{fig_vel_control}
\end{figure}

\subsubsection{Lateral Control}\label{lateral-control}

Lateral control aims to control the steering angle of the vehicle based
on trajectory generated by path planner. In our study we have developed
lateral controller based on Stanley's \citep{ventures2006stanley}
steering controller. Stanley steering controller is one of the popular
geometric path tracking approach used by Stanford university's
autonomous vehicle during DARPA grand challenge 2006. he closed-loop
geometric tracking controller helps vehicle to follow the desired path
generated by path planning. The controller is based on non-linear cross
track error \(C_{e}\) and heading error \(\theta_{e}\). Cross track
error is the lateral distance between front axle center point (P\_\{x\},
P\_\{y\}) and closest reference point (R\_\{x\}, R\_\{y\}) for which
exponential can be shown\citep{ventures2006stanley}. This term helps the
vehicle to steer towards the path to yield \(C_{e}\) as zero when its
following the path. The heading error \(\theta_{e}\) is calculated by
difference in vehicle heading \(\theta\) and reference heading
\(\theta_{r}\) of path at \((R_{x}, R_{y})\). This term error directly
apply to steering angle \(\delta\) with some steering angle limitation.

\begin{equation}
\label{eqn_heading_error}
\theta_{e} = \theta - \theta_{r}
%\caption{Heading error}
\end{equation}

When the cross track error \(C_{e}\) is non-zero, it adjusts \(\delta\)
such that the intended trajectory intersects the reference path tangent
from \((R_{x}, R_{y})\) at velocity in time \(V_{t}\) units from the
front axle center point \((P_{x}, P_{y})\). The resulting steering
control law by combining two terms are shown in equation
\ref{eqn_stanley_equation} below. The figure \ref{fig_stanley}
illustrates the geometric relationship of control parameters.

\begin{equation}
\label{eqn_stanley_equation}
\delta(t) = k_{h}\theta_{e} + tan^{^{-1}}\left (\frac{k_{c}C_{e}(t)}{V_{x}(t)}  \right )
\end{equation}

Fig ?? shows a control error during simulation. A Stanley controller can
be used for different type of course, such as parking, low speed, and
medium travels.

\begin{figure}[!t]
\centering
\includegraphics[width=2.5in]{stanley_2.pdf}
\caption{Stanley controller geometry}
\label{fig_stanley}
\end{figure}

\section{Health Monitoring}\label{health-monitoring}

Safety is very important in a critical mission project like autonomous
service vehicle. Not only that robust operation is required, but also
human safety must be prioritized in the system logic. Some components in
the system that needs attention are listed in Table \ref{table_risk_hm}.

\begin{table}[!t]
\renewcommand{\arraystretch}{1.3}
\caption{Risk in safety assessment for health monitoring module}
\label{table_risk_hm}
\centering
\begin{tabular}{c||c}
\hline\bfseries Component & \bfseries Risk\\
\hline
\hline Drive-By-Wire & Malfunction of motors\\
\hline
\hline GPS & GPS denied\\
\hline
\hline Camera & Cable loose\\
\hline
\end{tabular}
\end{table}

\section{Experimentation}\label{experimentation}

Fig. \ref{coms1_bare} shows the first vehicle platform that we first
retrofitted for the autonomous vehicle project.

\begin{figure}[!t]
\centering
\includegraphics[width=2.5in]{coms1_bare}
\caption{Experimentation of building vehicle platform.}
\label{coms1_bare}
\end{figure}

\subsection{Sensors}\label{sensors}

Placement of the sensors on the autonomous vehicle can be seen in Fig.
\ref{ipe_ex}, and their connection to the system can be summarized in
Fig. \ref{dia_ex}.

\begin{figure}[!t]
\centering
\includegraphics[width=2.5in]{ipe_ex.pdf}
\caption{Placement of sensors.}
\label{ipe_ex}
\end{figure}

\begin{figure}[!t]
\centering
\includegraphics[width=2.5in]{dia_ex.pdf}
\caption{Diagram of sensors connection.}
\label{dia_ex}
\end{figure}

\section{Conclusion}\label{conclusion}

That's all folks!

\section*{Acknowledgments}\label{acknowledgments}
\addcontentsline{toc}{section}{Acknowledgments}

Thanks the AV team in I2R that can accomplish many wonderful demos
within very short period of time. This leads to the conception of this
paper and many outstanding results that have been used in the demos.

                                    \renewcommand\refname{References}
                        \bibliography{papers}

    

\end{document}
